%%
%% This is file `sample-acmsmall-conf.tex',
%% generated with the docstrip utility.
%%
%% The original source files were:
%%
%% samples.dtx  (with options: `all,proceedings,bibtex,acmsmall-conf')
%% 
%% IMPORTANT NOTICE:
%% 
%% For the copyright see the source file.
%% 
%% Any modified versions of this file must be renamed
%% with new filenames distinct from sample-acmsmall-conf.tex.
%% 
%% For distribution of the original source see the terms
%% for copying and modification in the file samples.dtx.
%% 
%% This generated file may be distributed as long as the
%% original source files, as listed above, are part of the
%% same distribution. (The sources need not necessarily be
%% in the same archive or directory.)
%%
%%
%% Commands for TeXCount
%TC:macro \cite [option:text,text]
%TC:macro \citep [option:text,text]
%TC:macro \citet [option:text,text]
%TC:envir table 0 1
%TC:envir table* 0 1
%TC:envir tabular [ignore] word
%TC:envir displaymath 0 word
%TC:envir math 0 word
%TC:envir comment 0 0
%%
%%
%% The first command in your LaTeX source must be the \documentclass
%% command.
%%
%% For submission and review of your manuscript please change the
%% command to \documentclass[manuscript, screen, review]{acmart}.
%%
%% When submitting camera ready or to TAPS, please change the command
%% to \documentclass[sigconf]{acmart} or whichever template is required
%% for your publication.
%%
%%
\documentclass[acmsmall,screen,review,anonymous]{acmart}

\newcommand{\yang}[1]{{\color{red}{Yang: #1}}}
\newcommand{\bogdan}[1]{{\color{purple}{Bogdan: #1}}}

%%
%% \BibTeX command to typeset BibTeX logo in the docs
\AtBeginDocument{%
  \providecommand\BibTeX{{%
    Bib\TeX}}}

%% Rights management information.  This information is sent to you
%% when you complete the rights form.  These commands have SAMPLE
%% values in them; it is your responsibility as an author to replace
%% the commands and values with those provided to you when you
%% complete the rights form.
\setcopyright{acmlicensed}
\copyrightyear{2018}
\acmYear{2018}
\acmDOI{XXXXXXX.XXXXXXX}

%% These commands are for a PROCEEDINGS abstract or paper.
\acmConference[Conference acronym 'XX]{Make sure to enter the correct
  conference title from your rights confirmation emai}{June 03--05,
  2018}{Woodstock, NY}
%%
%%  Uncomment \acmBooktitle if the title of the proceedings is different
%%  from ``Proceedings of ...''!
%%
%%\acmBooktitle{Woodstock '18: ACM Symposium on Neural Gaze Detection,
%%  June 03--05, 2018, Woodstock, NY}
\acmISBN{978-1-4503-XXXX-X/18/06}


%%
%% Submission ID.
%% Use this when submitting an article to a sponsored event. You'll
%% receive a unique submission ID from the organizers
%% of the event, and this ID should be used as the parameter to this command.
%%\acmSubmissionID{123-A56-BU3}

%%
%% For managing citations, it is recommended to use bibliography
%% files in BibTeX format.
%%
%% You can then either use BibTeX with the ACM-Reference-Format style,
%% or BibLaTeX with the acmnumeric or acmauthoryear sytles, that include
%% support for advanced citation of software artefact from the
%% biblatex-software package, also separately available on CTAN.
%%
%% Look at the sample-*-biblatex.tex files for templates showcasing
%% the biblatex styles.
%%

%%
%% The majority of ACM publications use numbered citations and
%% references.  The command \citestyle{authoryear} switches to the
%% "author year" style.
%%
%% If you are preparing content for an event
%% sponsored by ACM SIGGRAPH, you must use the "author year" style of
%% citations and references.
%% Uncommenting
%% the next command will enable that style.
%%\citestyle{acmauthoryear}


%%
%% end of the preamble, start of the body of the document source.
\begin{document}

%%
%% The "title" command has an optional parameter,
%% allowing the author to define a "short title" to be used in page headers.
\title{A Study of Throttling Bugs in Server Applications}

%%
%% The "author" command and its associated commands are used to define
%% the authors and their affiliations.
%% Of note is the shared affiliation of the first two authors, and the
%% "authornote" and "authornotemark" commands
%% used to denote shared contribution to the research.
\author{Ben Trovato}
\authornote{Both authors contributed equally to this research.}
\email{trovato@corporation.com}
\orcid{1234-5678-9012}
\author{G.K.M. Tobin}
\authornotemark[1]
\email{webmaster@marysville-ohio.com}
\affiliation{%
  \institution{Institute for Clarity in Documentation}
  \city{Dublin}
  \state{Ohio}
  \country{USA}
}

\author{Lars Th{\o}rv{\"a}ld}
\affiliation{%
  \institution{The Th{\o}rv{\"a}ld Group}
  \city{Hekla}
  \country{Iceland}}
\email{larst@affiliation.org}

\author{Valerie B\'eranger}
\affiliation{%
  \institution{Inria Paris-Rocquencourt}
  \city{Rocquencourt}
  \country{France}
}

\author{Aparna Patel}
\affiliation{%
 \institution{Rajiv Gandhi University}
 \city{Doimukh}
 \state{Arunachal Pradesh}
 \country{India}}

\author{Huifen Chan}
\affiliation{%
  \institution{Tsinghua University}
  \city{Haidian Qu}
  \state{Beijing Shi}
  \country{China}}

\author{Charles Palmer}
\affiliation{%
  \institution{Palmer Research Laboratories}
  \city{San Antonio}
  \state{Texas}
  \country{USA}}
\email{cpalmer@prl.com}

\author{John Smith}
\affiliation{%
  \institution{The Th{\o}rv{\"a}ld Group}
  \city{Hekla}
  \country{Iceland}}
\email{jsmith@affiliation.org}

\author{Julius P. Kumquat}
\affiliation{%
  \institution{The Kumquat Consortium}
  \city{New York}
  \country{USA}}
\email{jpkumquat@consortium.net}

%%
%% By default, the full list of authors will be used in the page
%% headers. Often, this list is too long, and will overlap
%% other information printed in the page headers. This command allows
%% the author to define a more concise list
%% of authors' names for this purpose.
\renewcommand{\shortauthors}{Trovato et al.}

%%
%% The abstract is a short summary of the work to be presented in the
%% article.
\begin{abstract}

\end{abstract}

%%
%% The code below is generated by the tool at http://dl.acm.org/ccs.cfm.
%% Please copy and paste the code instead of the example below.
%%
\begin{CCSXML}
<ccs2012>
 <concept>
  <concept_id>00000000.0000000.0000000</concept_id>
  <concept_desc>Do Not Use This Code, Generate the Correct Terms for Your Paper</concept_desc>
  <concept_significance>500</concept_significance>
 </concept>
 <concept>
  <concept_id>00000000.00000000.00000000</concept_id>
  <concept_desc>Do Not Use This Code, Generate the Correct Terms for Your Paper</concept_desc>
  <concept_significance>300</concept_significance>
 </concept>
 <concept>
  <concept_id>00000000.00000000.00000000</concept_id>
  <concept_desc>Do Not Use This Code, Generate the Correct Terms for Your Paper</concept_desc>
  <concept_significance>100</concept_significance>
 </concept>
 <concept>
  <concept_id>00000000.00000000.00000000</concept_id>
  <concept_desc>Do Not Use This Code, Generate the Correct Terms for Your Paper</concept_desc>
  <concept_significance>100</concept_significance>
 </concept>
</ccs2012>
\end{CCSXML}

\ccsdesc[500]{Do Not Use This Code~Generate the Correct Terms for Your Paper}
\ccsdesc[300]{Do Not Use This Code~Generate the Correct Terms for Your Paper}
\ccsdesc{Do Not Use This Code~Generate the Correct Terms for Your Paper}
\ccsdesc[100]{Do Not Use This Code~Generate the Correct Terms for Your Paper}

%%
%% Keywords. The author(s) should pick words that accurately describe
%% the work being presented. Separate the keywords with commas.
\keywords{Do, Not, Us, This, Code, Put, the, Correct, Terms, for,
  Your, Paper}


\received{20 February 2007}
\received[revised]{12 March 2009}
\received[accepted]{5 June 2009}

%%
%% This command processes the author and affiliation and title
%% information and builds the first part of the formatted document.
\maketitle

\section{Introduction}

A server application can get overloaded when it is given more requests that it can handle,
resuling in server crash or degraded performance. To prevent server overloading, existing 
server applications often involve \emph{throttling} mechanisms, which delay or block
excessive incoming requests based on certain criteria.

While the idea seems straightforward, we observe implementing throttling mechanisms
properly is non-trivial, and bugs in their implementation can lead to severe consequences.
In this paper, we study XXX throttling related bugs from Y real-world applications.

\vspace{.05in}
\noindent
\textbf{Root cause.} 

\begin{itemize}

\item No throttling. XX bugs are caused by the fact that no throttling mechanisms
were incorporated. Since the applications studied are all mature open-source projects,
they all incorporate throttling mechanisms for common requests, and thus such no-throttling
bugs mostly occur for uncommon requests, such as recovery, decomission. etc.

\item Big request. XX bugs are caused by this problem, which means a single request can
become too big in certain scenarios, causing problems like out of memory. In these cases, throttling mechanisms
have no chance to resolve the issue. To solve this type of issues, the developer needs to
break the big request into multiple smaller ones and apply throttling on smaller requests.

\item Incorrect metric. XX bugs fall into this category. It means the throttling mechanism uses
one metric as the criteria to determine whether to throttle a request, but it should another. A
typical example is using number of requests or size of requests as the metric---both can be 
insufficient in certain scenarios.

\item Misconfiguration.

\item Buggy implementation.

\end{itemize}

\vspace{.05in}
\noindent
\textbf{Symptoms.}




\section{Study of real-world throttling bugs}

We consider the following generic model for throttling: The throttler maintains one or a few \emph{metrics}
to determine whether throttling should be applied. The metric can be resource utilization (e.g., CPU utilization, memory consumption, etc) 
of a certain component or even as simple as a counter. A task entering the system may increase the values
of these metrics and completing the request may decrease their values. Before a request enters the system,
the throttler will check metric values and maybe the potential impact of the request, and if allowing the request
to enter will increase metric values to be above certain thresholds, the throttler will either delay or reject the
request. Note that in practice, some simplified throttlers do not have all these steps. For example, some throttlers do not
have metrics and simply add a delay to every request.

In this paper, we study bugs happend in these steps.

\vspace{.1in}
\noindent
\textbf{Methodology.} We studied the bug reports of 11 applications, including Cassandra, Hadoop Common, 
HBase, HDFS, MapReduce, ZooKeeper, YARN, Ignite, Kafka, Flink, and Spark. We searched keywords including ``throttling'', ``overload'',
``rate limit'', ``out of memory'', etc, in their JIRA issues. We then read these issue reports manually to filter out those
that are not throttling bugs. We mainly filter out the following types:

\begin{itemize}

\item Generic performance issue. If an issue can cause overutilization of a certain resource, and the solution is to
optimize the application code to reduce overall resource consumption, then we consider it a generic performance issue
and do not include it in this study, because it does not affect the steps discussed above.

\item Permanent resource leak. We don't consider permanent resource leak, like memory leak, in this study. Throttling
may delay their effects, but does not fundamentally address this type of issues.

\item Lock contention. Lock contention can implicitly throttle incoming requests, and thus inefficient locking can cause throttling
issues. However, since lock contention is well studied, we exclude these issues from our study, except one category that
shows a similar pattern compared to other throttling bugs.

\item Load balancing. Throttling is often used together with load balancing (i.e., if a request is throttled, distribute it to
another server). However, if the solution of an issue improves load balancing but does not touch any of the steps
mentioned above, we do not include it in our study.

\end{itemize}

After fitlering, we get a total of 94 issues, including 17 from Cassandra, 9 from Hadoop Commons, 8 from HBase, 19 from HDFS, 9 from Ignite, 13 from Kafka, 
4 from MapReduce, 3 from Spark, 1 from YARN, 2 from ZooKeeper, 9 from Flink

\subsection{Symptoms}

Severity: Urgent, critial, major, normal, minor, blocker

Impact: Crash, low performance

Over or under utilization

\subsection{Root causes and solutions}

We classify the root causes into the following types. Note that some issues have more than
one root causes, so the sum of the cases below is larger than the total number of issues.

\vspace{.05in}
\noindent
\textbf{Missing throttling (26 issues).} These issues are caused by the fact that throttling
is completely missing for certain types of requests. Since the applications we studied are mature
open-source products, they have applied throttling to common requests in common scenarios, and thus
such missing throttling issues often happen for uncommon requests (e.g., recovery, decommission, etc)
and/or uncommon scenarios (e.g., a large number of affected nodes/data items, very large storage space on one node, etc).

The solution to these issues is straightforward --- add throttling.

\vspace{.05in}
\noindent
\textbf{Big task (14 issues).} These issues are caused by the fact that a single (uncommon) request
can cause resource over-utilization and thus there is no appropriate way to throttle it. In 13 of them,
the corresponding system does not throttle such requests and thus the singe request causes problems. In one
of them, the corresponding system throttles such a request but then the request can never get completed.

There are two common patterns in this category. First, a task can read a lot of data (e.g., recover a big
table), causing OOM. Second, a task can hold a global lock for a long time, blocking all other tasks.

The solution to this category is to split the big task into multiple smaller ones. For example, a task can
read data with an iterator, reading a portion of data at a time. A long task requiring global lock can
periodically release the lock, allowing other tasks to enter.

\vspace{.05in}
\noindent
\textbf{Incorrect metric (13 issues).} These issues are caused by using incorrect metrics during
throttling. This category is often caused by using an easy-to-compute metric to approximate
the real metric. For example, to limit the memory consumption of tasks, some systems
uses the number of tasks as the metric, assuming each task has the same memory
consumption, but this is problematic when some tasks consumes significantly more resource
than others.

The solution is to switch to the right metric.

\vspace{.05in}
\noindent
\textbf{Misconfiguration (6 issues).} These issues are caused by using a suboptimal default
threshold. While setting a different value is straightforward, some issues argue a static value
won't work and thus propose adaptive method to change the threshold values or at least
provide mechanisms to allow a user to change the value at run time.

\vspace{.05in}
\noindent
\textbf{Buggy logic (33 issues).} This category catches other bugs in the throttling logic.
There are a few common subcategories. First, some issues are caused by incorrect formula
to update metric or sleep/delay time. For example a few issues use integers to update and
store metric values, and in certain cases, some intermediate values can go below 1 and thus
become 0, causing unexpected behavior. 

Second, as mentioned at the beginning of this section,
the system should increase the metric value \emph{before} a task starts and decrease the metric 
value \emph{after} the task finishes. Some issues are caused by increasing metric value after
a task starts. If so, the throttling does not actually prevent resource over-utilization since the
resource is utilized before throttling takes effects. Similarly, if we decrease metric value before
a task fully completes, the metric value does not accurately capture resource utilization.

Other issues include a throttling configuration exists but is not applied due to missing implementation,
the limit is on uncompressed data but throttling is applied on compressed data (or vice versa), etc.

\vspace{.05in}
\noindent
\textbf{Post handling (4 issues).} These issues are not related to the core logic of throttling, but related
to how to handle the delayed or rejected tasks. On the one hand, those tasks should not be considered
failed and should be retried. On the other hand, they should be retried too frequently.



\end{document}
\endinput
%%
%% End of file `sample-acmsmall-conf.tex'.
